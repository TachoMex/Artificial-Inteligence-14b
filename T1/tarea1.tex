\section{Alan Mathison Turing}

	(1912-1954)

	Nación en Lodres, Inglaterra. Fue matemático y se le atribuye ser el padre de la computación. Sus principales estudios fueron
	sobre las máquinas de Turing, la inteligencia artificial y la teoría de la computación. Las máquinas de turing son modelos abstractos
	que modelan matematicamente los algoritmos que pueden tratar las computadoras, Pueden ser vistas como máquinas de estados finitos pero
	que tienen implementada una banda infinita de memoria, agregandoles las operaciones de moverse en la banda a un lado o al otro, leer y escribir el dato sobre el cual esta posicionado el cabezal de la banda infinita. Con base a esto atendió "Entscheidungsproblem" (problema de paro), que propuso Hilbert en 1900, donde demostro que no era posible tener un "meta-algoritmo" que dado un algoritmo y los datos de entrada nos pudiera decir si se ejecutaría en una cantidad finita de tiempo o no. También se formuló la pregunta: "¿Las máquinas piensan?" y desarrolló la prueba de Turing, que consiste en un Juez que sostendrá una conversación con una persona y una máquina a travez de una terminal. El juez podrá hacer preguntas aritmeticas o simplemente conversar. Si al final de la prueba el juez no sabe distinguir quien es quien, se dice que la maquina ha pasado el test de turing. 

\section{Allen Newell}
	
	Nació en San Franciso, EUA. Fue matemático e informático, pero entre sus áreas de investigación se enfocó a la inteligencia artificial y la psicología. Sus teorías eran sobre la forma en que los humanos atacaban los problemas y llegaban a distintas soluciones que no siempre eran las óptimas. El se dió cuenta que la mente humana no analizaba todos los escenarios posibles, si no que mas bien actuan de una forma instintiva, o mejor conocido en la computacion como heuristica, donde se obtienen buenas soluciones pero no siempre las mejores. Tambien se dió cuenta que las capacidades de procesamiento del cerebro humano no eran tan distantes de las computadoras, y con el paso de los años esto se volvio evidente al ver que existen computadoras con capacidades de computo muy superiores a las de los humanos. El quería enfocarse en la forma en que los humanos los resuelven, no le importaba llegar a soluciones incorrectas, pero si que fueran aproximadas a las humanas. El problema que buscaban resolver es conocido como el GPS (General Problem Solver). El GPS busca crear un programa que sea capaz de resolver cualquier tipo de problema. 
	

\section{Charles Forgy}

\section{Claude Shannon}

	Nació en Michigan, EUA. Fue matemático egresado del MIT. Su más grande trabajo fue la "Teoría matemática de las comunicaciónes", donde propuso múltiples formas de medir la comunicación, la medición de los datos entre muchas otras. Relativo a la inteligencia ariticial, el tenía dudas respecto a como podriamos lograr que las computadoras aprendan. Desarrolló un ratón eléctrico capaz de resolver laberintos.

\section{Frank Rossenblat}
	Se dedicó a muchas áreas de la ciencia, desde la psicología que fue su área original, las matematicas, computacion e incluso la música. En cuanto la inteligencia artificial, es considerado el fundador de las redes neuronales artificiales, Construyo una máquina llamada perceptrón que tenia conectada una cuadricula de 20x20 sensores fotoeléctricos a manera de ser una retina ocular, estas celdas le permitian a una computadora trabajar en forma de neurona, con estimulos recibidos de la luz externa y decidir si actuaban o no, como lo hacen las neuronas bilógicas. También construyo otro perceptron que se dedicaba al reconocimiento del habla.

\section{Herbert A. Simon}

\section{Referencias}
\begin{itemize}
	\item http://www.dma.eui.upm.es/historia\_informatica/Doc/Personajes/AlanTuring.htm 
	\item http://amturing.acm.org/award\_winners/newell\_3167755.cfm
	\item http://www.ieee.org/about/awards/tfas/rosenblatt.html
	\item http://www.dma.eui.upm.es/historia\_informatica/Doc/Personajes/ClaudeShannon.htm
	\item http://techchannel.att.com/play-video.cfm/2010/3/16/In-Their-Own-Words-Claude-Shannon-Demonstrates-Machine-Learning
	\item http://csis.pace.edu/~ctappert/srd2011/rosenblatt-contributions.htm
\end{itemize}